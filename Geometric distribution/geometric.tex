\documentclass[11pt]{article}

\usepackage[margin=1in, a4paper]{geometry}

\usepackage[utf8]{inputenc}

\usepackage{setspace}  % set spacing

\setstretch{1.25}  %stretch line space to multiple x

\usepackage[dvipsnames,table, xcdraw]{xcolor}
% If you use beamer only pass "xcolor=table" option, i.e. \documentclass[xcolor=table]{beamer}

\usepackage{shadowtext}

\usepackage{indentfirst} % indent the first paragraph of each section

\usepackage{float} %determine the position of figures in the document

\usepackage{tabularx} % extra features for tabular environment

\usepackage{amsmath, amsfonts, amssymb}  % improve math presentation

\usepackage{blkarray, bigstrut}

\usepackage{makecell}

\usepackage{mathtools}
\DeclarePairedDelimiter\ceil{\lceil}{\rceil}
\DeclarePairedDelimiter\floor{\lfloor}{\rfloor}



%++++++++++++++++++++++++++++++++++++++++++++++++++++++++++++++++

\usepackage{graphicx} % takes care of graphic including machinery

\graphicspath{ {../../logos/} }

%++++++++++++++++++++++++++++++++++++++++++++++++++++++++++++++++



\usepackage{caption}

\usepackage{subcaption}

\usepackage{tikz}
\usetikzlibrary{shapes}

\usepackage{lipsum,lmodern}

\usepackage[most]{tcolorbox}

\usetikzlibrary{trees}  %add binary trees

\usetikzlibrary {positioning}

\usepackage[final]{hyperref} % adds hyper links inside the generated pdf file

\hypersetup{
	colorlinks=true,       % false: boxed links; true: colored links
	linkcolor=blue,        % color of internal links
	citecolor=blue,        % color of links to bibliography
	filecolor=magenta,     % color of file links
	urlcolor=blue         
}

\usepackage{blindtext}

\usepackage{dirtytalk} %quotation marks


%********************************

%Bibliography

\usepackage[backend=biber,  style=alphabetic,  sorting=ynt]{biblatex}

\addbibresource{../../Mybib.bib}


%********************************


\usepackage{fancyhdr}

\pagestyle{fancy}

\fancyhf{}

\lhead{\footnotesize {Math notes: Geometric Distribution} }
\rhead{\footnotesize { } }
\cfoot{- \thepage \ -}

\title{\vspace{-90pt} 


%**************************************************

% Title Part
\textbf  {Peer-graded Assignment} }
\author{Cui, Xiaolong(Larry)}
\date{\today}


%*************************************************

\begin{document}

%\maketitle

\thispagestyle{plain}

%*************************************************

\begin{figure}[H] %[!tbp]
  \begin{subfigure}{0.3\textwidth}
    \includegraphics[width=\textwidth]{uol}
    %\caption{Flower one.}
    %\label{fig:f1}
  \end{subfigure}
  \hfill
  \begin{subfigure}{0.3\textwidth}
    \includegraphics[width=\textwidth]{goldsmiths}
    %\caption{Flower two.}
    %\label{fig:f2}
  \end{subfigure}
  %\caption{My flowers.}
\end{figure}

%****************************************************

\begin{flushright}

\footnotesize {Sept. 28th,  2021}
\end{flushright}

\begin{center}
\textbf{The Geometric Distribution} \\
\footnotesize {Study Notes $ | $ Written by Larry Cui}
\end{center}

%***************************************************

%\begin{abstract}
%\end{abstract}


%***************************************************

\setcounter{figure}{0}

\vspace{10pt}


Consider a series of independent trials and each has one of two outcomes,  success or failure.  If $p$ is the probability of success,  the geometric distribution means the probability at which the first success occurs.  Obviously,  a formula for the pdf of this distribution is:

\begin{tcolorbox}[
	enhanced, 
	width=\textwidth, 
	%center upper,
	fontupper=\normalsize,% \bfseries,
	drop fuzzy shadow southwest,
	boxrule=0.4pt,
	sharp corners,
	colframe=yellow!80!black,
	colback=yellow!10]
	
\textbf{\color{RoyalBlue} Geometric Distribution pdf} 
\[ 
p_X (k) = (1-p)^{k-1} p ,  \qquad k = 1, 2,  \dots
\]

\end{tcolorbox}



\section{\normalsize cdf of the geometric distribution:}
\[
\begin{aligned}
F_X(t) = P(X \leqslant t) 
	&= \sum _{k=1} ^t (1-p)^{k-1} p \\
	&= p \sum _{k=1} ^t (1-p)^{k-1} \\
	&= p \cdot \frac{1 - (1-p)^t}{1 - (1-p)} \\
	&= 1- (1-p)^t \\
\end{aligned}
\]



\section{\normalsize $E(X)$ of the geometric distribution:}
\[
\begin{aligned}
E(X) 
	&= \sum _{k=1} ^\infty k \cdot (1-p)^{k-1} p \\
	&= p \sum _{k=1} ^\infty k \cdot (1-p)^{k-1} \\
\end{aligned}
\]

Now take a look at the sum part,  we can tell it's the derivative of another series:
$$
\left[ \sum _{k=1} ^\infty - (1-p)^k \right] ' =  \sum _{k=1} ^\infty k \cdot (1-p)^{k-1}
$$
But the closed form of the formula within the brackets is $\displaystyle \frac{p-1}{p}$,  so
$$
 \sum _{k=1} ^\infty k \cdot (1-p)^{k-1} = \left( \frac{p-1}{p} \right)' = \frac{1}{p^2}
$$
and 
$$ E(X) = p \cdot \frac{1}{p^2} = \frac{1}{p} $$


\section{\normalsize Var$(X)$ of the geometric distribution: }

The key step to find variance is to find $E(X^2)$.  By the same technique,  we have
\[
\begin{aligned}
\left[ \sum _{k=1} ^\infty (1-p)^{k+1} \right]'' 
	&= \sum _{k=1} ^\infty (k+1)k \cdot (1-p)^{k-1} p \\
	&= \sum _{k=1} ^\infty k^2 \cdot (1-p)^{k-1} + k \cdot (1-p)^{k-1} \\
\end{aligned}
\]
The second term on the right side of the equation is $\displaystyle \frac{1}{p^2}$,  and we can also use a closed form to represent the left side,  so the  equation becomes
\[
\begin{aligned}
\left[ \frac{(1-p)^2}{p} \right]'' 
	&= \sum _{k=1} ^\infty k^2 \cdot (1-p)^{k-1} + \frac{1}{p^2} \\
\frac{2}{p^3}
	&= \sum _{k=1} ^\infty k^2 \cdot (1-p)^{k-1} + \frac{1}{p^2} \\
\end{aligned}
\]
so
\[
E(X^2) = p \sum _{k=1} ^\infty k^2 \cdot (1-p)^{k-1} = p \cdot \left( \frac{2}{p^3} - \frac{1}{p^2} \right) = \frac{2-p}{p^2} 
\]
and
\[
Var{X} = E(X^2) - E(X)^2 = \frac{2-p}{p^2}  - \frac{1}{p^2} = \frac{1-p}{p^2}
\]


\section{\normalsize \textit{mgf} of the geometric distribution: }
By definition,  the moment generating function of a distribution function is the expected value of $e^{tx}$:
\[
\begin{aligned}
M_k(t) = E(e^{tk}) 
	&= \sum _{k=1} ^\infty e^{tk} \cdot (1-p)^{k-1} p \\
	&= \frac{p}{1-p}  \sum _{k=1} ^\infty e^{tk} \cdot (1-p)^k \\ 
	&=  \frac{p}{1-p} \cdot \frac{e^t (1-p)}{1- e^t (1-p)} \\
M_k(t)	&= \frac {pe^t}{1- (1-p)e^t } \\
\end{aligned}
\] 







%++++++++++++++++++++++++++++++++++++++++

\clearpage

\printbibliography [title={Reference}]


%***********************************

\end{document}
