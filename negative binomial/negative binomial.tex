\documentclass[11pt]{article}

\usepackage[margin=1in, a4paper]{geometry}

\usepackage[utf8]{inputenc}

\usepackage{setspace}  % set spacing

\setstretch{1.25}  %stretch line space to multiple x

\usepackage[dvipsnames,table, xcdraw]{xcolor}
% If you use beamer only pass "xcolor=table" option, i.e. \documentclass[xcolor=table]{beamer}

\usepackage{shadowtext}

\usepackage{indentfirst} % indent the first paragraph of each section

\usepackage{float} %determine the position of figures in the document

\usepackage{tabularx} % extra features for tabular environment

\usepackage{amsmath, amsfonts, amssymb}  % improve math presentation
\allowdisplaybreaks

\usepackage{blkarray, bigstrut}

\usepackage{makecell}

\usepackage{mathtools}
\DeclarePairedDelimiter\ceil{\lceil}{\rceil}
\DeclarePairedDelimiter\floor{\lfloor}{\rfloor}



%++++++++++++++++++++++++++++++++++++++++++++++++++++++++++++++++

\usepackage{graphicx} % takes care of graphic including machinery

\graphicspath{ {../../logos/} }

%++++++++++++++++++++++++++++++++++++++++++++++++++++++++++++++++



\usepackage{caption}

\usepackage{subcaption}

\usepackage{tikz}
\usetikzlibrary{shapes}

\usepackage{lipsum,lmodern}

\usepackage[most]{tcolorbox}

\usetikzlibrary{trees}  %add binary trees

\usetikzlibrary {positioning}

\usepackage[final]{hyperref} % adds hyper links inside the generated pdf file

\hypersetup{
	colorlinks=true,       % false: boxed links; true: colored links
	linkcolor=blue,        % color of internal links
	citecolor=blue,        % color of links to bibliography
	filecolor=magenta,     % color of file links
	urlcolor=blue         
}

\usepackage{blindtext}

\usepackage{dirtytalk} %quotation marks


%********************************

%Bibliography

\usepackage[backend=biber,  style=alphabetic,  sorting=ynt]{biblatex}

\addbibresource{../../Mybib.bib}


%********************************


\usepackage{fancyhdr}

\pagestyle{fancy}

\fancyhf{}

\lhead{\footnotesize {Math notes: Negative Binomial} }
\rhead{\footnotesize { } }
\cfoot{- \thepage \ -}

\title{\vspace{-90pt} 


%**************************************************

% Title Part
\textbf  {Peer-graded Assignment} }
\author{Cui, Xiaolong(Larry)}
\date{\today}


%*************************************************

\begin{document}

%\maketitle

\thispagestyle{plain}

%*************************************************

\begin{figure}[H] %[!tbp]
  \begin{subfigure}{0.3\textwidth}
    \includegraphics[width=\textwidth]{uol}
    %\caption{Flower one.}
    %\label{fig:f1}
  \end{subfigure}
  \hfill
  \begin{subfigure}{0.3\textwidth}
    \includegraphics[width=\textwidth]{goldsmiths}
    %\caption{Flower two.}
    %\label{fig:f2}
  \end{subfigure}
  %\caption{My flowers.}
\end{figure}

%****************************************************

\begin{flushright}

\footnotesize {Sept. 29th,  2021}
\end{flushright}

\begin{center}
\textbf{The Negative Binomial Dsitribution} \\
\footnotesize {Study Notes $ | $ Written by Larry Cui}
\end{center}

%***************************************************

%\begin{abstract}
%\end{abstract}


%***************************************************

\setcounter{figure}{0}

\vspace{10pt}


Follow the logic of geometric distribution,  if we want to study the probability of $r^{th}$ success in $k^{th}$ of a series of trials,  it must be the case that $(r-1)$ success occur during the first $(k-1)$ trials and the $r^{th}$ happens on exactly the $k^{th}$ trial. 


\begin{tcolorbox}[
	enhanced, 
	width=\textwidth, 
	%center upper,
	fontupper=\normalsize,% \bfseries,
	drop fuzzy shadow southwest,
	boxrule=0.4pt,
	sharp corners,
	colframe=yellow!80!black,
	colback=yellow!10]
	
\textbf{\color{RoyalBlue} Negative Binomial Distribution pdf} 
\[ 
p_X (k) = \binom {k-1}{r-1} p^r (1-p)^{k-r} ,  \qquad k = r, r+1,  \dots
\]

\end{tcolorbox}

If we let $X$ be the sum of independent variables $X_1,  X_2,   \dots,   X_r$,  and let $X \to \infty$,  the negative binomial can be interpreted as $r$ successes happen one after another,  and each of which follows the geometric distribution model.  This perspective won't give us a better form to the pdf,  but will greatly simplify the calculation of expected values. 


\begin{tcolorbox}[
	enhanced, 
	width=\textwidth, 
	%center upper,
	fontupper=\normalsize,% \bfseries,
	drop fuzzy shadow southwest,
	boxrule=0.4pt,
	sharp corners,
	colframe=yellow!80!black,
	colback=yellow!10]
	
\textbf{\color{RoyalBlue} E(X), Var(X) and mgf} \qquad Let $ \displaystyle p_X (k) = \binom {k-1}{r-1} p^r (1-p)^{k-r},  \quad k=r, r+1, \dots$ Then 
\quote {
a.  $\displaystyle M_X (t) = M_{X_1}(t) M_{X_2}(t) \dots M_{X_r}(t) = \left[ \frac {pe^t}{1-(1-p)e^t} \right]^r $ \\
b.  $\displaystyle E(X) = E(X_1) + E(X_2) + \cdots + E(X_r) = \frac{r}{p} $ \\
c.  $\displaystyle Var(X) = Var(X_1) + Var(X_2) + \cdots + Var(X_r) = \frac{r(1-p)}{p^2}$
}

\end{tcolorbox}




\section*{\normalsize Algebraic Approach: the sum of pdf}

There's an algebraic approach to the $E(X)$ and Var$(X)$, which centres around the verification of the pdf of the function.  Let's take a look at the following lemma:

\begin{tcolorbox}[
	enhanced, 
	width=\textwidth, 
	%center upper,
	fontupper=\normalsize,% \bfseries,
	drop fuzzy shadow southwest,
	boxrule=0.4pt,
	sharp corners,
	colframe=yellow!80!black,
	colback=yellow!10]
	
\textbf{\color{RoyalBlue} lemma (a)} 
\[ 
\binom {k+m}{m} = \binom {k+m-1}{m} + \binom {k+m-1}{m-1}
\]

\end{tcolorbox}

\textbf{Proof:} a simple expansion of the combinations gives us result.  Furthermore,  we can also see this equation by some intuition: if you want pick $m$ items from $k+m$ pool,  you can first split the pool into one item and the other.  The final pick consists of two scenarios: 1) excluding this one item: $\displaystyle \binom {k+m-1}{m}$; and 2) including this one item: $\displaystyle \binom {k+m-1}{m-1}$. 
\[
\begin{aligned}
\binom {k+m-1}{m} + \binom {k+m-1}{m-1} 
	&= \frac{(k+m-1)!}{m! (k-1)!} + \frac{(k+m-1)!}{(m-1)! k!} \\
	&= \frac{(k+m-1)! k + (k+m-1)! m}{m! k!} \\
	&= \frac{(k+m)!}{m!k!}
\end{aligned}
\]

\textbf{Intermediate function:} consider the function
\[
f_m(z) = \sum _{k=0} ^\infty \binom {k+m}{m} z^k
\]
Use lemma (a),  we can expand it as
\[
f_m(z) = \sum _{k=0} ^\infty \binom {k+m-1}{m-1} z^k + \sum _{k=0} ^\infty \binom {k+m-1}{m} z^k
\]
Because  $\displaystyle \binom {m-1}{m} = \frac{(m-1)(m-2) \cdots (m-m)}{m! (-1)!} = 0 $ by definition,  we can discard the $k=0$ item from the second term.  A little re-arrangement give the following form for the second term:
\[
\sum _{k=1} ^\infty \binom {k+m-1}{m} z^k = z \sum _{k=1} ^\infty \binom {k+m-1}{m} z^{k-1} = z f_m(z) 
\]
so
\[
f_m(z) = f_{m-1}(z) + z f_m(z) \quad \Rightarrow \quad f_m(z) = \frac{f_{m-1} (z)}{1-z}
\]
But since $\displaystyle f_0 (z) = \sum ^\infty _{k=0} \binom {k}{0} z^k = \frac{1}{1-z}$ ,  we know $\displaystyle f_m(z) = \frac{1}{(1-z)^{m+1} } $.\\


To verify the sum of the pdf equals to 1,  we use the above function to get (here we let $m=r-1,  k'=k-r$):
\begin{equation}
\begin{aligned}[t] 
\sum _{k=r} ^\infty \binom {k-1}{r-1} p^r (1-p)^{k-r} 
	&= p^r \cdot  \sum _{k'=0} ^\infty \binom {k' + m}{m}  (1-p)^{k'} \\
	&= p^r \cdot f_m (1-p) \\
	&= p^r \cdot \frac{1}{(1-1+p)^r} = 1 & \color{RoyalBlue} \text{\footnotesize Remark: } r=m+1 \\
\end{aligned}
\end{equation}



Now we can go on with $E(X)$:
\begin{equation}
\begin{aligned}[t]
E(X) 
	&= \sum _{k=r} ^\infty \binom {k-1}{r-1} k p^r (1-p)^{k-r} \\
	&= p^r \cdot  \sum _{k'=0} ^\infty (k'+r)  \binom {k' + m}{m}  (1-p)^{k'}   \\
	&=  p^r \cdot  \sum _{k'=0} ^\infty r  \binom {k' + r}{r}  (1-p)^{k'} && \color{RoyalBlue}  \text{\footnotesize Remark: } m=r-1\\
	&= r p^r \cdot \frac{1}{(1-1+p)^{r+1} } = \frac{r}{p}  \\
\end{aligned}
\end{equation}


In order to find Var$(X)$,  we need a little trick here: find $E[X (X+1)]$ first.  As above,  we start by lay out the formula:
\[
\begin{aligned}
E[X(X+1)] 
	&= \sum _{k=r} ^\infty k(k+1) \binom {k-1}{r-1}  p^r (1-p)^{k-r} \\
	&= p^r \cdot  \sum _{k'=0} ^\infty (k'+r+1)(k' + r)  \binom {k' + m}{m}  (1-p)^{k'}  && \color{RoyalBlue}  \text{\footnotesize Remark: } k=k'+r\\
	&=  p^r \cdot  \sum _{k'=0} ^\infty (k'+r+1)(k' + r)  \binom {k' + r -1}{r-1}  (1-p)^{k'}  && \color{RoyalBlue}  \text{\footnotesize Remark: } m=r-1 \\
	&= r (r+1) p^r   \cdot  \sum _{k'=0} ^\infty  \binom {k' + r +1}{r+1}  (1-p)^{k'}  \\
	&= r (r+1) p^r \cdot \frac{1}{(1-1+p)^{r+2} } \\
	&= \frac{r(r+1)}{p^2} \\
\end{aligned}
\]
so we have
\[
\text{Var}(X) = E[X(X+1)] - E(X)^2 - E(X)= \frac{r(r+1)}{p^2} - \frac{r^2}{p^2} - \frac{r}{p}=\frac{r(1-p)}{p^2}
\]


\textbf{A second way:}

We can also find the solution to Equation (1) without having recourse to intermediate function $f_m(z)$.  First of all,  we re-arrange the expected value formula as:
\[
\begin{aligned}
E(X) 
	&= \sum _{k=r} ^\infty \binom {k-1}{r-1} k p^r (1-p)^{k-r} \\
	&= p^r \cdot  \sum _{k'=0} ^\infty (k'+r)  \binom {k' + r-1}{r-1}  (1-p)^{k'} && \color{RoyalBlue}  \text{\footnotesize Remark: } k'=k-r\\
	&= r p^r \cdot \sum _{k'=0} ^\infty  \binom {k' + (r+1) -1}{k'}  (1-p)^{k'} \\
	&= r p^r \cdot \sum _{k'=0} ^\infty  \binom {k' + (r+1) -1}{k'}  (-1)^{k'} (p-1)^{k'} \\
\end{aligned}
\]


We need Taylor/Maclaurin series for help here.  Recall


\begin{tcolorbox}[
	enhanced, 
	width=\textwidth, 
	%center upper,
	fontupper=\normalsize,% \bfseries,
	drop fuzzy shadow southwest,
	boxrule=0.4pt,
	sharp corners,
	colframe=yellow!80!black,
	colback=yellow!10]
	
\textbf{\color{RoyalBlue} The Maclaurin series for $f(x)$}  

wherever it converges, can be expressed as: 
\[ 
f(x) = f(0) + f'(0)x + \frac{f''(0)}{2!}x^2 + \frac{f^{(3)} (0)}{3!}x^3 + \cdots +  \frac{f^{(k)} (0)}{k!}x^k + \cdots
\]

and for $f(x) = 1/(1+x)^n$: 
\[
\frac{1}{(1+x)^n} = 1 - nx + \frac{(-n)(-n-1)}{2!} x^2 + \cdots = \sum _{k=0} ^\infty \binom {n+k-1} {k} (-1)^k x^k
\]


\end{tcolorbox}


We can see that the last part of $E(X)$ is exactly the sum of Maclaurin series with $n=r+1$ and $x=p-1$,  so
\[
E(X) = r p^r \cdot \frac{1}{(1 + p -1)^{r+1}} = \frac{r}{p} \\
\]













%++++++++++++++++++++++++++++++++++++++++

\clearpage

\printbibliography [title={Reference}]


%***********************************

\end{document}
